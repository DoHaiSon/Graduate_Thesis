\clearpage
\phantomsection

\addcontentsline{toc}{chapter}{Danh mục ký hiệu và chữ viết tắt}
\chapter*{Danh mục   ký hiệu và chữ viết tắt}
{\renewcommand{\arraystretch}{1.4}
{\fontsize{12}{13}\selectfont
\begin{longtable}{|C{0.8cm}|>{\raggedright}p{5.4cm}|p{8cm}|}
\hline
\multicolumn{3}{|l|}{\textbf{Danh mục ký hiệu}}\\
\hline
\hline
\textbf{STT} & \textbf{Ký hiệu} & ~\hfill\textbf{Giải thích}\hfill~\\
\hline
1&in thường&Vô hướng\\
\hline
2&in thường, đậm&Vector\\
\hline
3&in hoa, đậm&Ma trận\\
\hline
4&$\times$&Phép nhân vô hướng\\
\hline
5&$\lambda$&Bước sóng của tín hiệu\\
\hline
6&$\phi$&Góc phương vị\\
\hline
7&$\theta$&Gốc ngẩng\\
\hline
8&$\mu_{\mathbf{x}}$&Giá trị trung bình của $\mathbf{x}$\\
\hline
9&$\mathcal{E}$&Phép tính trung bình\\
\hline
10&$\mathbf{a}$&Vector đáp ứng mảng\\
\hline
11&$\mathbf{C}_{\mathbf{x}} = \mathcal{E} \{(\mathbf{x}_i - \mathbf{\mu}_i)(\mathbf{x}_j - \mathbf{\mu}_j)\}$&Ma trận hiệp phương sai của $\mathbf{x}$\\
\hline
12&$\mathbf{E}$&Vector riêng\\
\hline
13&$\mathbf{E}_{\mathbf{s}}$&Ma trận chứa vector riêng của không gian con tín hiệu\\
\hline
14&$\mathbf{E}_{\mathbf{n}}$&Ma trận chứa vector riêng của không gian con tạp âm\\
\hline
15&$\mathbf{\Lambda}$&Giá trị riêng\\
\hline
16&$\mathbf{\Lambda}_{\mathbf{s}}$&Ma trận chứa giá trị riêng của không gian con tín hiệu\\
\hline
17&$\mathbf{\Lambda}_{\mathbf{n}}$&Ma trận chứa giá trị riêng của không gian con tạp âm\\
\hline
18&$\mathbf{I}$&Ma trận đơn vị	\\
\hline
19&K&Số mẫu thu thập\\
\hline
20&$\mathbf{n}$&Vector tạp âm\\
\hline
21&$\mathbf{R}_{\mathbf{x}} = \mathcal{E}\{\mathbf{x}(t)\mathbf{x}(t)^H\}$&Ma trận tương quan của $\mathbf{x}$\\
\hline
22&$\mathbf{s}$&Vector nguồn tín hiệu đến\\
\hline
23&$\mathbf{w}$&Vector trọng số phức\\
\hline
24&$\mathbf{x}$&Vector tín hiệu thu thập tại mỗi phần tử anten\\
\hline
25&$(\mathrm{x}_m, \mathrm{y}_m)$&Vị trí của phần tử anten thứ $m$ trong không gian 2D\\
\hline
26&$\mathbf{A}$&Ma trận chứa các vector đáp ứng mảng\\
\hline
27&$H$&Phép biến đổi Hermitan\\
\hline
28&$T$&Phép chuyển vị\\
\hline

\end{longtable}
}}
\newpage
{\renewcommand{\arraystretch}{1.2}
{\fontsize{12}{13}\selectfont
\begin{longtable}{|C{0.8cm}|p{1.7cm}|>{\raggedright}p{5.5cm}|p{5.75cm}|}
\hline
\multicolumn{4}{|l|}{\textbf{Danh mục chữ viết tắt}}\\
\hline
\hline
%\multirow{2}{*}{\textbf{STT}} & \textbf{Chữ viết tắt} & ~\hfill\textbf{Giải thích tiếng Anh}\hfill~&~\hfill\textbf{Giải thích tiếng Việt}\hfill~\\
\multirow{2}{*}{\textbf{STT}} & ~\hfill\textbf{Chữ}\hfill~ &\multicolumn{1}{c|}{\multirow{2}{*}{\textbf{Giải thích tiếng Anh}}}&\multicolumn{1}{c|}{\multirow{2}{*}{\textbf{Giải thích tiếng Việt}}}\\
& ~\hfill\textbf{viết tắt}\hfill~ & & \\
\hline
1&ADC&Analog Digital Converter&Bộ chuyển đổi tương tự sang số\\ 
\hline
2&AM&Amplitude Modulation&Điều chế biên độ\\
\hline
3&BCCH&Broadcast Control Channel&Kênh quảng bá\\
\hline
4&CBF&Conventional Beamforming method&Phương pháp tạo búp sóng truyền thống\\ 
\hline
5&CDMA&Code Division Multiple Access&Đa truy cập phân chia theo mã\\
\hline
6&CFO&Carrier Frequency Offset&Độ lệch tần số sóng mang\\
\hline
7&CRB&Cramer Rao Bound &Đường bao Cramer Rao\\
\hline
8&DAC&Digital Analog Converter&Bộ chuyển đổi số sang tương tự\\
\hline
9&DC&Direct Current&Dòng điện một chiều\\
\hline
10&DOA&Direction of Arrival&Hướng sóng đến\\
\hline
11&DVB-T&Digital Video Broadcasting – Terrestrial&Phát sóng truyền hình số mặt đất\\
\hline
12&ESPRIT&Estimation of Signal Parameters Via Rotational Invariance Techniques&Thuật toán ước lượng các tham số của tín hiệu thông qua kỹ thuật quay bất biến\\ 
\hline
13&FFT&Fast Fourier Transform&Biến đổi Fourier nhanh\\ 
\hline
14&FIR&Finite Impulse Response&Đáp ứng xung hữu hạn\\ 
\hline
15&FM&Frequency Modulation&Điều chế tần số\\
\hline
16&FPGA&Field-programmable Gate Array&Mảng cổng logic có thể lập trình\\ 
\hline
17&GFSK&{\small Gaussian Frequency Shift Keying}&Khóa dịch tần số Gaussian\\ 
\hline
18&GPIO&General Purpose Input / Output&Cổng vào/ra đa chức năng\\ 
\hline
19&GPP&General Purpose Processor&Bộ xử lý đa năng\\ 
\hline
20&GRC&GNU Radio Companion&Giao diện người dùng GNU Radio\\
\hline
21&GSM&Global System for Mobile Communications&Hệ thống thông tin di động toàn cầu\\ 
\hline
22&I2C&Inter-Intergrated Circuit&Bus giao tiếp giữa các IC\\
\hline
23&IFFT&Inverse Fast Fourier Transform&Biến đổi Fourier ngược\\
\hline
24&IIR&Infinite Impulse Response&Đáp ứng xung vô hạn \\
\hline
25&IoT&Internet of Things&Internet vạn vật\\    
\hline
26&IQ&In-phase and Quadrature components&Thành phần cùng pha và vuông pha\\
\hline
27&JTAG&Joint Test Action Group&Cổng thử nghiệm lập trình\\
\hline
28&JTRS&Joint Tactical Radio System&Hệ thống vô tuyến chiến thuật\\    
\hline
29&LNA&Low Noise Amplifier&Khuếch đại tạp âm thấp\\
\hline
30&LO&Local Oscillator&Dao động nội\\
\hline
%17&LS-ESPRIT&Least Squares ESPRIT&\\    
%\hline

31&MIMO&Multiple Input - Multiple Output&Nhiều đầu vào và nhiều đầu ra\\    
\hline
32&ML&Maximum Likelihood&Thuật toán giống nhất cực đại\\    
\hline
%20&MMITS& Modular Multifunction Information Transfer Systems&Hội nghị về truyền thông năm 1996\\    
%\hline
33&MODE&Method of Direction Estimation&Phương pháp xác định hướng đến\\    
\hline
34&MUSIC&MUtiple SIgnal Classification&Thuật toán phân lớp nhiều tín hiệu\\    
\hline
35&MVDR&Minimum Variance Distorsionless Response&Đáp ứng không méo phương sai tối thiểu\\    
\hline
36&NBFM&Narrow-band Frequency Modulation&Điều chế tần số băng hẹp\\    
\hline
37&OFDM&Orthogonal Frequency Division Multiplexing&Ghép kênh phân chia theo tần số trực giao\\    
\hline
38&OTT&Out Of Tree Module&Khối không chính thức của GNU Radio\\
\hline
39&PA&Power Amplifier&Khuếch đại công suất\\    
\hline
40&PLL&Phase Locked Loop&Vòng bám pha\\    
\hline
41&PN&Pseudo-random Noise&Nhiễu giả ngẫu nhiên\\    
\hline
42&PPM&Part per Million&Phần triệu\\
\hline
43&PSK&Phase Shift Keying&Điều chế pha nhị phân\\
\hline
44&QAM&Quadrature Amplitude Modulation&Điều chế biên độ cầu phương\\
\hline
45&RF-IF&Radio Frequency - Intermediate Frequency&Bộ chuyển đổi từ tần số cao xuống trung tần\\        
\hline
46&RMSE&Root Mean Square Deviation&Sai số bình phương trung bình căn\\
\hline
47&SDR&Software Define Radio&Vô tuyến định nghĩa bằng phần mềm\\    
\hline
48&SMB&System Management Bus&Bus quản lý hệ thống\\
\hline
49&SNR&Signal to Noise Ratio&Tỷ số tín hiệu trên tạp âm\\    
\hline
%33&TLS-ESPRIT&Total Least Squares ESPRIT&\\    
%\hline
50&TS&Transport Stream&Giao vận trực tuyến\\
\hline
51&ULA&Uniform Linear Array&Mảng thẳng cách đều\\    
\hline
52&USRP&Universal Software Radio Peripheral&Thiết bị ngoại vi vô tuyến\\    
\hline
53&VCTCXO&Voltage Controlled Temperature Compensated Crystal Oscillators&Bộ tạo dao động tinh thể bù nhiệt độ\\
\hline
54&WAV&Waveform Audio File Format&Tệp âm thanh dạng sóng\\
\hline
55&WBFM&Wide-band Frequency Modulation&Điều chế tần số băng rộng    \\
\hline    
\end{longtable}
}
}
			