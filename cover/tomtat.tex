\clearpage
\phantomsection

\addcontentsline{toc}{chapter}{Tóm tắt}
\chapter*{\fontsize{13}{13}\selectfont{Tóm tắt}}
\fontsize{12}{12}\selectfont{
\noindent\textbf{Tóm tắt:}
Hệ tìm phương, hay còn gọi là tìm hướng sóng đến (DOA), luôn đóng vai
trò quan trọng trong các ứng dụng: thông tin, định vị, giám sát, dẫn đường, tìm kiếm cứu nạn,... Với sự phát triển vượt bậc của xử lý tín hiệu, hệ tìm phương sử dụng thuật toán MUSIC cho phép xác định hướng sóng đến của nhiều nguồn phát ở độ chính xác, độ phân giải cao, cấu trúc mảng anten tùy ý, điều mà các hệ tìm phương truyền thống không thể. Các thiết bị vô tuyến định nghĩa bằng phần mềm SDR kết hợp với phần mềm GNU Radio được mong đợi sẽ trở thành công nghệ thống trị truyền thông vô tuyến trong tương lai. Bắt kịp xu thế đó, trong khóa luận này sẽ tập trung tìm hiểu về phần cứng và phần mềm cho SDR: BladeRF x115, lập trình các khối  phục vụ DOA trên GNU Radio. Qua đó triển khai thuật toán tìm phương MUSIC trên hệ BladeRF để kiểm nghiệm khả năng đồng bộ và kết quả ước lượng hướng sóng đến của hệ.

\vspace{0.5cm}
\noindent\textit{\textbf{Từ khóa:}} \textit{DOA, MUSIC, SDR, GNU Radio.}
}